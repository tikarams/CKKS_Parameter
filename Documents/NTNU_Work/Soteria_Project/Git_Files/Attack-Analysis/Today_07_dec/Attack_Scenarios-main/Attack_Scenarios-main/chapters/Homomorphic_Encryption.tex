% !TeX root = ../sameplaper.tex
% !TeX spellcheck = en_US

\chapter{Homomorphic Encryption}
\setcounter{section}{0}


\section{Security Notion}
We will briefly introduce security notions of cryptography that readers come across when studying homomorphic encryption schemes. Security notions of homomorphic encryption schemes are the same as that of symmetric and asymmetric cryptographic encryption schemes. We briefly review security notions here for a thorough security notion of cryptography; we refer readers to \cite{katz2020introduction}.


\subsection{Security Model}
The notion of an adversary in cryptography is a modeled one. The actual adversaries in the real world are referred to as an attacker. In cryptography, attackers are usually assumed to be computationally bounded and run in probabilistic polynomial time. However, in some cases, computationally unbounded adversaries are also considered. An adversary can corrupt parties involved in the protocol. Based on how parties are corrupted or behave while executing a protocol, adversaries can be classified into Honest-But-Curious (HBC) and Malicious adversaries. The difference between the two adversary models will be clear from the definition below.

\begin{definition} \cite{goldreich2009foundations} (Honest-But-Curious Adversary). An honest-but-curious (HBC) adversary is a legitimate participant in a communication protocol who will not deviate from the defined protocol but can gather all information shared for protocol execution from the legitimately received messages.
\end{definition}

The HBC adversary model is a reasonably weak adversarial model. Still, this model is sufficient to ensure no unintended information leakage from the protocol. Chosen plaintext attack (CPA) is an example of an HBC adversary where the adversary can observe the honestly generated ciphertexts. An HBC adversary model is also known as a semi-honest adversary or passive adversary.


\begin{definition} \cite{goldreich2009foundations} (Malicious Adversary). A malicious adversary is a legitimate participant in a communication protocol who takes active steps to disrupt the execution of the protocol (send messages that differs from those specified by the protocol), or can gather legitimate messages shared for protocol execution (he can use it later or shared with the other dishonest parties).
\end{definition}

A malicious adversary model is a strong adversary model. Chosen ciphertext attack (CCA) is an example of a malicious adversary model where the adversary can ask for any ciphertext of its choice regardless of valid or invalid. Malicious adversaries are also known as an active adversaries.


\subsection{Semantic Security}
Correctly defining security of a protocol or an encryption scheme is non-trivial task. The idea is that the adversary should not learn any partial information about the plaintext from the ciphertext. It is usually formalized using semantic security. Loosely speaking, semantic security means whatever can be efficiently computed from the ciphertext can be efficiently computed given only the length of the plaintext. Note that semantic security does not rule out the possibility of inferring the length of the plaintext from the ciphertext. However other than the length of the plaintext the ciphertext should not yield anything about the plaintext.

Semantic security is complex and difficult to work with. Thus an equivalent definition called \textit{indistinguishability} is used. To be precise an encryption scheme is said to be semantically secure if and only if it has indistinguishable encryptions \cite{goldreich2009foundations}. Below we will briefly review indistinguishability.


\subsection{Indistinguishability}
Indistinguishability is based on an experiment involving an adversary passively observing a ciphertext and then trying to guess the encrypted message. It can be played as a game where an adversary $\mathcal{A}$ first samples two messages $\mu_1$ and $\mu_2$ and sends them to the challenger. The challenger selects one of the message $\mu_1$ or $\mu_2$ with equal probability and encrypts and sends the encrypted message to the adversary. If $\mathcal{A}$ outputs a ``guess" as to which of the two messages was encrypted; $\mathcal{A}$ succeeds if the guess is correct. There are standard models of indistinguishability available for different attack scenarios as defined below.
% INDistinguishability under Chosen Plaintext Attack (IND-CPA), INDistinguishability under Chosen Cihpertext Attack (IND-CCA), INDistinguishability under adaptively Chosen Cihertext Attack (IND-CCA2), INDistinguishability under Cihpertext Verification Attack (IND-CVA) etc as defined below.

\subsubsection{IND-CPA} An encryption scheme is considered IND-CPA secure if no adversary can distinguish encryption of two messages $\mu_1$ and $\mu_2$ better than probability $\frac{1}{2}$. Here the adversary has black-box access to an encryption oracle, and he can ask for encryption of the plaintext message of his choice. The idea of this security definition is that even if the adversary has access to the encryption oracle, he can not deduce any information about the plaintext by simply observing ciphertext.

IND-CPA is equivalent to the property of semantic security, and many cryptographic proofs use these definitions interchangeably. It is the standard security requirement for an encryption scheme. State-of-the-art homomorphic encryption schemes, regardless of whether integers-based or LWE-based, provides IND-CPA security by direct reduction to an intractable problem.

Notice that any scheme that achieves IND-CPA security is secure in the presence of an HBC adversary. Also, by this security definition, any deterministic encryption scheme cannot be secure as the adversary can ask for encryption of messages $\mu_1$ and $\mu_2$, compare it with the challenge ciphertext and produce the result with probability 1.

\subsubsection{IND-CCA} An encryption scheme is said to be IND-CCA secure if no adversary can distinguish encryption of two messages $\mu_1$ and $\mu_2$ better than $\frac{1}{2}$ probability even if the adversary has black-box access to the encryption and decryption oracle before the challenge ciphertext $c$ is published. Alternatively, the adversary should not learn any information about the plaintext message by looking into the ciphertext even after having black-box access to the encryption and decryption oracle. IND-CCA is a stronger security notion than IND-CPA.

It has been shown in \cite{loftus2011cca,zhang2012cca,chenal2015key} that almost all somewhat homomorphic encryption schemes proposed till now are vulnerable to non-adaptative chosen ciphertext attacks (CCA1), revealing the decryption key. In fact it is obvious that no malleable encryption scheme can achieve full IND-CCA security. More discussion about the same can be obtained in \cite{bellare1999non}.

In general, all somewhat homomorphic encryption schemes based on LWE or its ring variant, and also those based on integers, rely on a search-to-decision reduction. This basically means that being able to decrypt a homomorphic ciphertext is equivalent to recovering the private key, which itself reduces to solving some intractable problem in the worst case. Suppose this feature allows elegant IND-CPA security proofs. In that case, it also means that any (CCA1) decryption oracle breaks the scheme just by following the steps of the search-to-decision reduction.




\subsubsection{IND-CCA2} IND-CCA2 security is a stronger security notion than that of IND-CPA and IND-CCA. An encryption scheme is said to be IND-CCA2  secure if no adversary can distinguish encryption of two messages $\mu_1$ and $\mu_2$ better than $\frac{1}{2}$ probability, even if he has black-box access to the encryption and decryption oracle even after obtaining a challenge ciphertext $c$ except that he can not ask for decryption of the challenge ciphertext $c$.

No homomorphic primitive can be IND-CCA2 secure since their main feature is to be malleable.

\subsubsection{IND-CVA}Standard security definition alone may not be sufficient when we study encryption schemes from a deployment point of view. We must see beyond the encryption primitive and consider a broader view of the overall system deployement. In case of homomorphic encryption following questions makes sense when considered deployment.
\begin{enumerate}
    \item Can the cloud have some meaningful information when it behaves as an active attacker ? or
    \item Can a user acting as a passive or an active attacker can perform some attacks when it gets access to the decryption oracle?
\end{enumerate}



In some encryption schemes, extracting the secret key itself might be possible if the attacker gets access to the decryption oracle like the one presented by Li et al. \cite{li2021security} in the case of approximate encryption scheme \cite{cheon2017homomorphic}. This was possible due to secret key leakage in the decrypted ciphertext. However, these types of attacks are not possible in the exact homomorphic encryption scheme. Still, we need to consider possible attack scenarios when deploying a homomorphic encryption scheme for real-world applications.

Indistinguishability under Ciphertext Verification Attacks (IND-CVA) allows the adversary to access to both encryption oracle and ciphertext verification oracle. This notion of security is slightly stronger than IND-CPA but much weaker than IND-CCA \cite{hu2009ciphertext}.
% It many be used to attack the deployment scenario by modifying few bits of the ciphertext and observing the clients behaviour.
% \paragraph{Safe-error and reaction attack}: The security of homomorphic encryption primitive are not sufficient to use homomorphic encryption schemes in real-world applications. It must be studied by considering the broader view or construction of how the schemes have been used for the application.
IND-CVA focus on the overall deployment or system behavior. These are the attacks that are performed by an active attacker. This type of attack is feasible when the cloud behaves as an active attacker. These attacks focus on and use the overall deployment behavior when a small error is injected into the computed result. The devastating nature of this type of attack has been studied in \cite{chillotti2016attacking} and suggested some technique to avoid such attacks viz.

The mitigation techniques of such attacks is to use signatures along with the encrypted data. Compute and check signatures before processing the data for its integrity.



\section{FHE Security Assumption} Deployment of encryption schemes for real-world applications needs some standards and security parameterization. The same goes for homomorphic encryption schemes. Standards of homomorphic encryption have been studied and presented in the report \cite{albrecht2021homomorphic}. The report presents the currently known state-of-the-art lattice attacks on lattice-based encryption schemes and recommends a wide selection of parameters to be used for homomorphic encryption for various security levels. The report also describes the additional features of HE schemes which make them useful in different application scenarios.

Below we have briefly reviewed hard problems backing homomorphic encryption schemes followed by known attacks and their computation complexity.

\subsection{Hard problems}
In the literature to design homomorphic encryption schemes, different hard problems have been used, viz. NTRU ($N^{th}$ degree Truncated polynomial Ring Units) problem, AGCD (Approximate Greatest Common Divisor) problem, LWE (Learning With Error) problem and it's variant R-LWE (Ring Learning With Error) problem. However, not all these hard problems give rise to equally efficient encryption schemes. LWE and R-LWE-backed encryption schemes are usually more efficient from practical considerations. They produce comparatively smaller size ciphertext than the others. However, encryption schemes backed by an overstretched \cite{albrecht2016subfield} variant of the NTRU computational problem are shown to be vulnerable to subfield lattice attacks \cite{albrecht2016subfield,cheon2016algorithm}; thus, they are no longer used in practice. For the same reason, we will not introduce NTRU problem here.


\subsubsection{Learning With Error (LWE) Problem}
Introduced by Regev \cite{regev2009lattices}, learning with errors (LWE) problem finds many application in cryptography. The LWE problem can be defined as follows.

\begin{definition}For security parameter $\lambda$, let $n = n(\lambda)$ be an integer dimension, let $q = q(\lambda) \geq 2$ be an integer modulus, and let $\chi = \chi(\lambda)$ be an error distribution over $Z$. The decisional version of $LWE_{n,q,\chi}$ problem is to distinguish the following two distributions:

    In the first distribution, one samples $(a_i, b_i)$ uniformly from $Z^{n+1}_q$.

    In the second distribution, one first draws $s \leftarrow Z^n_q$ uniformly and then samples $(a_i, b_i) \in Z^{n+1}_q$ by sampling $a_i \leftarrow Z^n_q$ uniformly, $e_i \leftarrow \chi$, and setting $b_i = <a_i,s> + e_i$.

    The search version of the $LWE_{n,q,\chi}$ problem corresponds to the recovery of the secret vector $s$ given polynomially many samples from the second distribution.

    The $LWE_{n,q,\chi}$ assumption is that the $LWE_{n,q,\chi}$ problem is infeasible.
\end{definition}

The error distribution $\chi$ can be either a discrete Gaussian distribution over the integers, a continuous Gaussian distribution rounded to the nearest integer, or other distributions supporting small integers.

Security of the LWE problem is established by quantum reduction from hard lattice problems such as GapSVP and SIVP \cite{regev2009lattices}. %$An efficient solution to the LWE problem implies a quantum algorithm for GAPSVP and SIVP.
In fact, for the hardness of LWE problem when the error distribution $\chi = \mathcal{D}_{Z,\alpha q}$ where $\alpha \cdot q \geq 2 \sqrt{n}$, it has been established that the search version of LWE is at least as hard as quantumly approximating certain worst-case problems on $n$-dimensional lattices to within $\tilde{O}(n/\alpha)$ factors \cite{regev2009lattices}. Moreover, for similar parameters and large enough $q$, search-LWE is at least as hard as classically approximating the decision shortest vector problem and variants \cite{peikert2009public}. For moduli $q$ that are sufficiently `smooth' (i.e., products of small primes), the decision form of LWE is at least as hard as the search form \cite{peikert2009public,regev2009lattices}. It has been also established that the decision-LWE remains hard even if the secret $s$ is chosen from the error distribution $\chi$, rather than uniformly at random \cite{micciancio2009lattice,applebaum2009fast}.  Although, binary, and ternary LWE is easier than general LWE still properly choosen parameters can provide binary-LWE same securty as that of the standard LWE problem \cite{chen2020concrete,bai2014lattice}.



\subsubsection{Ring Learning With Error (R-LWE) Problem}

Introduced by Lyubaskevsky, Peikert and Regev \cite{lyubashevsky2013ideal} the ring learning with errors (R-LWE) problem finds many applications in cryptography. Here we provide a simplified special-case version of the problem that is easier to understand and work with \cite{brakerski2011fully},\cite{regev2010learning}.

\begin{definition}
    For security parameter $\lambda$, let $f(x) = x^d+1$ where degree $d = d(\lambda)$ is a power of 2. Let modulus $q = q(\lambda) \geq 2$ be an integer. Let $R = Z[x]/(f(x))$ and let $R_q = R/qR$. Let $\chi = \chi(\lambda)$ be a distribution over $R$. The decisional R-LWE$_{d,q,\chi}$ problem is to distinguish the following two distributions:

    In the first distribution, one samples $(a_i, b_i)$ uniformly from $R^2_q$.

    In the second distribution, one first draws secret $s \leftarrow R_q$ uniformly and then samples $(a_i, b_i) \in R^2_q$ by sampling $a_i \leftarrow R_q$ uniformly, $e_i \leftarrow \chi$, and sets $b_i = <a_i, s> + e_i$.

    The search version of the R-LWE$_{n,q,\chi}$ problem corresponds to the recovery of the secret $s$ given polynomially many samples from the second distribution.

    The R-LWE$_{d,q,\chi}$ assumption is that the R-LWE$_{d,q,\chi}$ problem is infeasible.
\end{definition}

The efficiency of the R-LWE problem over LWE problem is that though in R-LWE each noisy product $b \approx a\cdot s$ gives $n$ simultaneously pseudorandom values over $\mathbb{Z}_q$ unlike LWE, still the cost of generating it is significantly small. In general polynomial multiplication can be performed in $O(n\log{n})$ scalar operations and in parallel depth $O(\log{n})$, using  Fast Fourier Transform (FFT) or its variants. Also, in most applications, $(a,b)\in R_q \times R_q$ from the R-LWE distribution can replace $n$ samples $(\textbf{a},b)\in \mathbb{Z}_q^n\times \mathbb{Z}_q$ from the standard LWE distribution; reducing the public key size by a factor of $n$. This dramatically reduces the public key size of homomorphic encryption schemes.

Security of the R-LWE problem is established by quantum reduction from the approximate Shortest Vector Problem (SVP) (in the worst case) on ideal lattices in $R$ to the search version of the R-LWE problem.

\subsubsection{Approximate Greatest Common Divisor (AGCD) problem} Introduced by Howgrave-Graham in \cite{howgrave2001approximate}, AGCD is an alternative way of designing homomorphic encryption schemes. The problem can be defined as %The first alternative design was proposed by van Dijk, Gentry, Halevi and Vaikuntanathan \cite{van2010fully} in 2011. The AGCD problem can be defined as
\begin{definition}
    Let $\rho,\eta,\gamma$ and $p$ be integers such that $\gamma > \eta > \rho > 0$ and $p$ is an $\eta$-bit prime number. The distribution $D_{\gamma,\rho}(p)$, whose support is $[0, 2^{\gamma}-1]$ is defined as
    \begin{align*}
        D_{\gamma,\rho}(p) & := \{Sample\ q \leftarrow [0, 2^{\gamma}/p[ \ and\ r \leftarrow ]-2^{\rho},2^{\rho}\ [\ Output\ x := pq + r\}
    \end{align*}
    The ($\rho, \eta, \gamma$)-AGCD problem is the problem of finding $p$, given polynomially many samples from $D_{\gamma,\rho}(p)$.

    The ($\rho,\eta,\gamma$)-decisional-AGCD problem is the problem of distinguishing between $D_{\gamma,\rho}(p)$ and $\mathcal{U}([0, 2^{\gamma}[)$.
\end{definition}

AGCD problem is believed to be hard even in the presence of quantum computers. In fact, when the parameters ($\rho,\eta,\gamma$) are chosen properly, the best known attacks against it run in exponential time \cite{galbraith2016algorithms}. Moreover, when $p,q_i$ and $r_i$ are sampled from a specific distributions, then the AGCD problem becomes at least as hard as the LWE problem \cite{cheon2015fully}.

The first homomorphic encryption scheme over the integers designed using the hardness of the AGCD problem was proposed by Dijk et al. \cite{van2010fully} in 2010. The original work is followed by the subsequent work few of which includes \cite{cheon2013batch,coron2014scale,cheon2015fully,benarroch2017fhe,pereira2020efficient} to name a few, however, the list goes on. Each of these works tried to improve the performance of a homomorphic encryption scheme designed using the AGCD problem.



\subsubsection{NTRU Problem}
Introduced by Hoffstein, Pipher, and Silverman \cite{NTRUJHJPJHS,NTRUEncrypt,hoffstein2006ntru} in 1996, NTRU is an efficient lattice based encryption scheme. The problem is be defined as
\begin{definition}
    Let $q \geq 2$ an integer and $\gamma > 0$ a real number. A $(\gamma, q)$-NTRU instance is an element $h \in R_q=\mathbb{Z}_q[x]/(x^d+1)$(ring of integers of the degree-$d$ cyclotomic number field) such that there exists $(f,g) \in R^2 \setminus \{(0,0)\}$ with $g\cdot h = f \ mod\ q$ and $||f||,||g|| \leq \sqrt{q}/\gamma$. The pair
    $(f,g)$ is called a trapdoor of the NTRU instance $h$.

    For $t \geq 1$ and $\gamma$ and $q$ as above, a $(\gamma,q,t)$-tuple-NTRU instance is a tuple $(h_i)_{i\leq t} \in R_q$ such that there exists $((f_i)_{i\leq t},g) \in R^{t+1} \setminus \{(0,\cdots,0)\}$ with $g\cdot h_i=f_i$ mod $q$ and $max_i ||f_i||,||g|| \leq \sqrt{q}/\gamma$.


    % \begin{definition}
    % Let $q \geq 2$, $\gamma > 0$ and $D$ a distribution over $(\gamma, q)$-NTRU instances. A $(D, \gamma, q)$-NTRU probabilistic polynomial time algorithm (with respect to $\log{q}$ and $\log{\Delta_K}$) sampling triples $(h, f, g) \in R_q \times R^2$ such that
    % \begin{itemize}
    %     \item The marginal distribution of $h$ is $D$,
    %     \item $(f,g) \ne (0,0)$ and $||f||,||g||\leq \sqrt{q}/\gamma$,
    %     \item $g\cdot h= f \mod q$
    % \end{itemize}
    % \end{definition}

    For a  distribution $D$ over $(\gamma, q)$-NTRU instances. The $(D, \gamma, q)$ decisional NTRU problem ($(D, \gamma, q)$-dNTRU for short) is the problem of distinguishing  between the samples from $D$ and from $U(R_q)$.

    The $(D, \gamma, \gamma',q)$ $(\gamma \geq \gamma' > 0)$ search NTRU problem is the problem of computing a pair $(f,g) \in R^2 \setminus \{(0,0)\}$ such that $g \cdot h = f\ mod\ q$ and $||f||, ||g|| \leq \sqrt{q}/\gamma'$ given an input $h$ sampled from $\mathcal{D}$.

\end{definition}

Note that any $(\gamma, q)$-NTRU instance is also a $(\gamma',q)$-NTRU instance for any $\gamma'\leq\gamma$.



Initially, the NTRU problem lacked security proof. In 2011, Stehl\'{e} and Steinfeld \cite{stehle2011making} showed how to modify NTRU to reduce security to standard problems in ideal lattices. In 2012, LTV \cite{lopez2012fly} proposed a fully homomorphic encryption scheme based on NTRU using overstretched \cite{albrecht2016subfield}(when $q$ is super-polynomial in $d$) variant of the NTRU problem. The work of BLLN \cite{bos2013improved} follows the work of LTV with a more enhanced encryption scheme.

Recently, Albrecht et al. \cite{albrecht2016subfield} showed that overstretched NTRU is vulnerable to subfield lattice attacks, and it is believed to be over of NTRU-based homomorphic encryption schemes. Later, it was defined precisely how much is called overstretched in NTRU by Ducas et al. \cite{ducas2021ntru,bonte2023final}. Thus security of NTRU-based homomorphic encryption can be established by properly selecting parameters restricting it from overstretching.



\section{HE Parameterization}
The security of a homomorphic encryption scheme depends on the underline hard problems and their parameters. The hardness of homomorphic encryption schemes is based on the hardness of LWE or a Ring-LWE problem. Thus security of the HE scheme is defined using a ring $R$ with dimension $n$, ciphertext modulus $q$, error distribution $\chi$, and a choice of secret distribution $s$.
\subsection{Ring} : Ring $R$ to use in a homomorphic encryption scheme is a power of $2$-cyclotomic polynomial as $R=Z[x]/(x^n+1)$, where $n$ is a power of $2$. For these rings, no known attacks are available that work better than attacks on the LWE problem.

\subsection{Error distribution} In homomorphic encryprion error is usually sampled from Discrete Gaussian distribution with standard deviation $\sigma$=$\frac{8}{\sqrt{2\pi}}$. The reason behind the use of Discrete Gaussian distribution is motivated by the theoretical security reduction provided by \cite{lyubashevsky2013ideal}.However, those theoretical guarantees do not apply to fixed small error widths such as $\sigma$=$\frac{8}{\sqrt{2\pi}}$, and would instead require that the error width grow with the square root of the dimension of the lattice, $\sqrt{n}$ . None of the known attacks appear to take advantage of the shape of the error distribution, only the width; however, the analysis of the security levels given below relies on running time estimates which assume that the shape of the error distribution is Discrete Gaussian. For that reason we continue to assume that the error is chosen from a Discrete Gaussian distribution of fixed small width. The width is chosen to be small for practical performance reasons, and is justified by the concrete running times of known attacks with those error widths. However, over time if attacks improve or new attacks are found, the width of the error may need to be increased in practice.

\subsection{Secret key distribution} For efficiency reasons, in practice we often choose the secret from a non-uniform distribution, such as the ternary distribution, which means to select uniformly from $\{-1,0,1\}$. In the recommended parameters given below, we will present tables for three choices of secret-distribution:{uniform, error, and ternary}. We will not present tables for sparse secrets because we do not suggest standardizing the case of sparse secret due to significantly better known attacks.


\subsection{Recommended parameters for HE}
Table \ref{Tab:parameters_HE} shows the list of parameter values suggested by the homomorphic encryption standar to retain
different security levels.


\begin{table}[ht]
    \centering
    \begin{tabular}{|c|c|c|c|c|c|}
        \hline
        N                        & security level & $\log q$ & uSVP  & dec   & dual  \\
        \hline
        \multirow{3}{4em}{1024}  & 128            & 25       & 132.6 & 165.5 & 142.3 \\
                                 & 192            & 17       & 199.9 & 284.1 & 222.2 \\
                                 & 256            & 13       & 262.6 & 423.1 & 296.6 \\
        \hline
        \multirow{3}{4em}{2048}  & 128            & 51       & 128.6 & 144.3 & 133.4 \\
                                 & 192            & 35       & 193.5 & 231.9 & 205.2 \\
                                 & 256            & 27       & 257.1 & 327.8 & 274.4 \\
        \hline
        \multirow{3}{4em}{4096}  & 128            & 101      & 129.6 & 137.4 & 131.5 \\
                                 & 192            & 70       & 193.7 & 213.6 & 198.8 \\
                                 & 256            & 54       & 259.7 & 295.2 & 270.6 \\
        \hline
        \multirow{3}{4em}{8192}  & 128            & 202      & 129.8 & 130.7 & 128.0 \\
                                 & 192            & 141      & 192.9 & 202.5 & 196.1 \\
                                 & 256            & 109      & 258.3 & 276.6 & 263.1 \\
        \hline
        \multirow{3}{4em}{16384} & 128            & 411      & 128.2 & 129.5 & 129.0 \\
                                 & 192            & 284      & 192.0 & 196.8 & 193.7 \\
                                 & 256            & 220      & 257.2 & 265.8 & 260.7 \\
        \hline
        \multirow{3}{4em}{32768} & 128            & 827      & 128.1 & 128.7 & 128.4 \\
                                 & 192            & 571      & 192.0 & 194.1 & 193.1 \\
                                 & 256            & 443      & 256.1 & 260.4 & 260.4 \\
        \hline
    \end{tabular}
    \caption{The differnt parameters shown in the table represents the following: $N$ is the ring dimension, security level is the bit security provided by the parameters equivalent to that of AES bit security. $\log q$ is the number of bits in the modulus $q$. $uSVP$ represents the bit security against unique shortest vector attack, $dec$ represents the bit security against decoding attack and $dual$ represnts the bit security against dual attack.}
    \label{Tab:parameters_HE}
\end{table}



\section{Library}
There are several research groups working around the world in this new technology and have implemented the general-purpose homomorphic encryption schemes mentioned above in the form of  libraries for applications and general-purpose use. Few of these library are shown in Table \ref{Tab:Crypt_library}. One thing that is common in these general-purpose libraries is that are based on the ring learning-with-error (RLWE) problem, and many of them displayed common choices for the underlying rings, error distributions, and other parameters.

\begin{table}
    \begin{tabular}{ |p{2.5cm}||p{3.5cm}|p{3cm}|p{3cm}|}
        \hline
        Library          & Deveploper        & Developed In & Implements \\
        \hline
        Concrete \cite{} & ZAMA              & Rust         & AFG        \\
        CuFHE    \cite{} & AX                & 1            & ALA        \\
        HEAAN    \cite{} & DZ                & 1            & DZA        \\
        HElib    \cite{} & AL                & 1            & ALB        \\
        Lattigo  \cite{} & AS                & 1            & ASM        \\
        NFLlib   \cite{} & AO                & 1            & AGO        \\
        OpenFHE  \cite{} & formerly PALISADE & 1            & 2          \\
        PALISADE \cite{} & AO                & 1            & AGO        \\
        SEAL     \cite{} & AD                & 1            & AND        \\
        TFHE     \cite{} & AO                & 1            & AGO        \\
        \hline
    \end{tabular}
    \caption{Table showing the HE libraries available}
    \label{Tab:Crypt_library}
\end{table}




