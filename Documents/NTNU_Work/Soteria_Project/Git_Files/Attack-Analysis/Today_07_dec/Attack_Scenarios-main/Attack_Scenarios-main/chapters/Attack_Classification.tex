% !TeX root = ../sameplaper.tex
% !TeX spellcheck = en_US

\chapter{Attack Classification}
\setcounter{section}{0}


Categorization of attacks
Attacks can be mainly categorized into

\section{Passive attacks}
\begin{itemize}
    \item Here all the components follow the protocol however, they passively keep track of all steps followed and any private information that can be gained from it.
    \item To avoid such attacks the system must be designed that leaks as minimum information as possible.
\end{itemize}

\section{Active attacks}
\begin{itemize}
    \item In this attack setting the components can deviate from the protocol and behave independently resulting into even wrong computation.
    \item These types of attacks are usually hard to take care of.
\end{itemize}

\section{Side Channel Attacks}
\begin{itemize}
    \item These are the attacks that are based on the side information that are gathered when the protocol or the algorithm is implemented and used.
    \item Timing information, power consumption electromagnetic leaks and sound analysis are the examples of side information’s that are exploited to facilitate the side-channel attacks.
\end{itemize}

\section{Denial of Service attack}
\begin{itemize}
    \item These are the attacks meant to shut down a machine or network, making it inaccessible to its intended users.
    \item DoS attacks are usually accomplished by flooding the target with traffic or sending it information that triggers a crash.
    \item In both instances, the DoS attack deprives legitimate users of the service or resource they expected.
\end{itemize}

\section{Insider attack}
\begin{itemize}
    \item These are the attackers who have access to the inside leakage and can mount attack based on the information they have.
\end{itemize}

\section{Combination of attacks}
\begin{itemize}
    \item In this attack techniques different attack gets combined and uses the combined information to extract the unauthorized information.
\end{itemize}

