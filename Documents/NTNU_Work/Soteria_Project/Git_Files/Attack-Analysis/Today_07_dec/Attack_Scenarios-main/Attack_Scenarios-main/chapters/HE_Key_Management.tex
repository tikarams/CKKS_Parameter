% !TeX root = ../sameplaper.tex
% !TeX spellcheck = en_US

\chapter{HE Key Management}
\setcounter{section}{0}

\section{Key Management System}

Recommendation for key management compiled by NIST \cite{KMSV5},\cite{KMS} provides cryptographic key-management guidance. In our scenario, we can use the recommendations they mentioned for HE key management for the smooth management of HE keys. In our deployment, we assume that encryption keys are generated by \textcolor{red}{Database owner} as he is the sole owner of the database and wants to ensure that his database's privacy never gets breached. However, he can hire a third party for the key generation. The third-party generates the HE keys based on the required security level, which the KMS later manages.

The KMS is the one responsible for management of FHE keys stored. These keys are later requested by Client Service Portal and TEE computing the decryption and decoding operations. Keys stored in the KMS have to provide security from all forms of attackers. Here are some guidlince to store keys for overall secure FHE query computation.

\begin{enumerate}
    \item If same public key is used for all users in the institution then encryption key must be the private key corresponding to the public key of the same institution. If this is the case then all institution must have the same public key. If different public key has been used for different institutions then what encryption key or private key has been used to encrypt the data?

    \item Is it that different public key has been generated for the same private keys?

    \item If so what will be its effect viz. how many public keys we can generate without revealing or leaking any information about the private key to the Query service portal behaving as an active attacker?
\end{enumerate}





\newpage
\section{Questions}
\begin{enumerate}
    \item What happens when the homomorphic encryption keys are compromized? written below
    \item Attack to the HE system -------- User adds more errors to the query for decryption failure, corrupted cipher text, corrupted result, effective ways of communication, encoding is known to the attacker.
          ----- Quantify the risk for all the above possibilities? Not necessary to address may come from the protocol itself
    \item How the system behaves when one entity crashes? No side informations leak
    \item Effect of behaviour of one component to the next component when they are in pipeline? By design it will stop working
          \textcolor{blue} {\item What happens when more than one entity comes together to perform an attack?}
    \item How to avoid Service Query Portal performing active attack say changes the digital signature of the user by its own signature? \textcolor{blue} {Can he perform such attack? May be restricted using centralized key storage}
          \textcolor{blue} {\item What will be the overhead of using TEE in the cloud computation as mentioned above in the encrypted cloud computation? Present in literature mentioned in ETH paper
    \item What information is leaked by the feedback system? Done
    \item Can a genuine user mount an attack to get secret information (plain data or the secret key)? Done (It can perform decryption failure attack adding more error in the query encryption phase)
    \item What side information we can collect from the system? Done computations are performed in TEE. Thus, hopefully no information is leaked.
    \item Attacks on API gateway? Done
    \item Underline messaging system? Done
    \item Corrupt the ciphertext in the client side that is not decryptable-correctly resulting into decryption failure? Already addressed using digital signature}
\end{enumerate}


\newpage
% FHE security assumptions and security parameterization
\begin{center}
    \underline{\textbf{Questions}}
\end{center}
\begin{enumerate}
    \item What infomration can we collect using Fault injection attack ?
    \item If assumptions do not hold true, then the list of potential threats that a bed actors can do
    \item If assumptions do not hold then possible attacks in CPA and CCA scenario
    \item Use of decryption oracle to decrypt result, can adversary also gets the decrypted result before decoding. In such scenarios, what should we do?
    \item Use a little bit of error and check the behavior of the client or the user. In doing so, what information can be learned?
    \item Explore ciphertext injection attack.
    \item Preservation of depth of the computation.
    \item Explore the equilibrium of the cost of preserving and the cost of attacking.
    \item Make a table and show what may cause what attack and its vulnerabilities.
    \item Consider FHE computation separately, separate from key generation and ciphertext generation. Vulnerability assumptions for the HBC FHE computation setting. What HBC server can do to the intermediate data?
\end{enumerate}

\begin{itemize}
    \item Corrupt the computation
    \item Corrupt the values
    \item Deviation from the computed values by known quantity to know the user behavior and use this information to get side channel information
\end{itemize}





\newpage
\section{HE secret keys compromized after few computation}

The HE secret key may get compromised after a while even if the secret is only sent for query result decryption in the TEE. Mitigating such attack is possible by avoiding sending of actual secret key for query result decryption. Use key switching technique to obtain a switching key and a new secret key corresponing to the actual secret key. Send the key switching key to the ;of the computed using the of the secret key private key switching key to the TEE FHE query computation. TEE FHE query computation, after query computation, multiplies the calculated result with the key switching key and sends the computed query result to the encrypted result store, which is later decrypted in the TEE environment using the newly switched key obtained from the KMS. This way, a new key is generated using the key-switching technique after every few computations. The newly generated keys are sent for query result decryption which gets changed after every few calculations, thus restricting the actual secret key compromisation attack.

\section{Attack due to collaboration of more than one entity}
\subsection{Client Service Portal and KMS} When these two entities come together, they do not bring any new advantage to perform any new attacks in addition to those mentioned above.

\subsection{User and KMS} When these two-entity come together for attacks they can perform all attacks mentioned above. In addition, the user now has advantage of access to all institution/user keys using these keys the user can ask services of different kinds. Such situations are normal when KMS gets compromised.

\textcolor{blue} {However, this problem can be avoided by storing keys encrypted using some attribute-based encryption scheme so that those keys can be decrypted for use by the authorized institution/user only.}

\subsection{User, Client Service Portal and KMS} Addition of Client Service Portal to the combination of User and KMS do not give any new advantage to lunch any new attacks other than the one mentioned above. When the entity User, Client Service Portal and KMS come together any unauthorized user can access the authorized user’s Institution/user keys and get services as a legitimate user without getting caught. However, this attack is same as that of combination of User and KMS and can be restricted by using the same technique as that of mentioned above.

\subsection{Service Query Portal and User Information Store} When Service Query Portal and User Information Store gets combine, they cannot lunch any new attack. If they try to perform any malleability they will get caught as they must send a feedback message and query signature is centrally handled.

\subsection{Client Service Portal and Service Query Portal} No new attacks can be lunched if these two entities combine. This is because if they change anything in the query, then either they need to send feedback to the user or they need to send the query to the next step where query signature is checked against the query. Thus, in either case they will be caught.

\subsection{User and User Information Store} If the user is an authorized user, then this gives no advantage to the user. However, if the user is an unauthorized user, then this results in giving access to the authorized users information. However still the unauthorized user may not be able to ask any query if the Client Service Portal is 2 tier authentication system i.e., to login to the Client Service Portal requires both password and OTP.  Thus, this gives overall no advantage to perform any new attack.

\subsection{KMS and User Information Store} They can not lunch any new attack even if they collaborate. This is because they need to send feedback to the user if they perform anything that is not allowed to perform based on the protocol. If they do not follow the protocol, then they will get caught by the user or in the next step of the computation.

\subsection{User, Client Service Portal, KMS, Service Query Portal and User Information Store} Collaboration of Service Query Portal and User Information Store to the User, Client Service Portal, KMS as mentioned above gives no new advantage as in the newly added entities only the authentication gets performed. As the user has access to the KMS he can use the genuine user’s information from KMS rather than using any unauthorized user information, so this added two entities gives no advantage than that of the combination of User, Client Service Portal and KMS.

\subsection{FHE Query Engine and Service Registry} If they combine, then they can change the service type check however they can’t deny the query of an authentic user as if they do so then they must send feedback mentioning the reason for the deny and in that process, they will get caught. Also, they can’t perform any malleability attack on the query as if they do so then they will get caught in the query integrity verification phase performed in the TEE.

\subsection{User and Service Registry} If user collaborates with the Service Registry, then he can perform any query request regardless of whether he has access to perform such query request or not and can get the service. \textcolor{blue} {It is hard to restrict such attacks viz. when service registry gets compromised.}

\subsection{User, Client Service Portal, KMS, Service Query Portal, User Information Store, FHE Query Engine and Service Registry} If all mentioned entities combine then any user can access the service using the genuine user’s credentials. However still the combination of the mentioned entities cannot get any information about the secret key or the database.

\subsection{User and Service KMS} When the user has access to the KMS he will not get any new information from it as the KMS is encrypted with the attestation key of the TEE. Thus, no new attacks can be lunched.

\subsection{User and Database} The database is encrypted with the attestation key of the TEE thus no new information can be gained, or no new attacks can be lunched.

\textcolor{blue} {All possible cases of combinations of different entities have been covered above and other than combination of USER and Service Registry in other cases no new attacks can be lunched.}
