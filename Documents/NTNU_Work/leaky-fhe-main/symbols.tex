%%%NEW symbols
\newcommand{\eq}{\texttt{eq}}
\newcommand{\sign}{\texttt{sign}}
\newcommand{\Szero}{\ensuremath{0_{\rings}}}
\newcommand{\Sone}{\ensuremath{1_{\rings}}}
\newcommand{\LL}{\mathcal{L}} % Set of labels
\newcommand{\LLplus}{\ensuremath{\mathcal{L}^{+}}} % Set of labels
\newcommand{\LLminus}{\ensuremath{\mathcal{L}^{-}}} % Set of labels
\newcommand{\fmin}{\ensuremath{\phi_\LL}} % Label map




\newcommand{\inputtt}{\texttt{inp}}
\newcommand{\outputtt}{\texttt{out}}
\newcommand{\accepttt}{\texttt{accept}}
\newcommand{\rejecttt}{\texttt{reject}}
\newcommand{\pp}{\texttt{pp}}
\newcommand{\monoid}{\ensuremath{\mathbf{M}}}
%\newcommand{\TNCpolys}{\ensuremath{R\langle\vvXX\rangle}}
%\newcommand{\TNCpolys}{\ensuremath{R\lfloor\vvXX\rceil}}
%\newcommand{\TNCpolys}{\ensuremath{R\abs{\vvXX}}}
%\newcommand{\TNCpolys}{\ensuremath{R\langle\hspace{-1.1mm}\abs{\vvXX}\hspace{-1.1mm}\rangle}}
\newcommand{\TNCpolys}{\ensuremath{R\llbracket\vvXX\rrbracket}} %Easy to confuse with power series
\newcommand{\NCApolys}{\ensuremath{R_A[\vvXX]}}

\newcommand{\TM}{T_{\mathcal{M}_n}} %Complexity for matrix multiplication.

%%GKR: Wiring predicates, etc
\newcommand{\add}[1][]{\mathtt{add}^{#1}}
\newcommand{\mult}[1][]{\mathtt{mult}^{#1}}
\newcommand{\tadd}[1][]{\widehat{\mathtt{add}}^{#1}}
\newcommand{\tmult}[1][]{\widehat{\mathtt{mult}}^{#1}}
\newcommand{\Vi}[1][i]{\ensuremath{\hat{V}^{(#1)}}}
\newcommand{\VL}{\ensuremath{\hat{V}_L}}
\newcommand{\VR}{\ensuremath{\hat{V}_R}}
\newcommand{\VLi}[1][i]{\ensuremath{\hat{V}_L^{(#1)}}}
\newcommand{\VRi}[1][i]{\ensuremath{\hat{V}_R^{(#1)}}}
\newcommand{\VLdot}{\ensuremath{\dot{V}_L}}
\newcommand{\VRdot}{\ensuremath{\dot{V}_R}}
\newcommand{\VLdoti}[1][i]{\ensuremath{\dot{V}_L^{(#1)}}}
\newcommand{\VRdoti}[1][i]{\ensuremath{\dot{V}_R^{(#1)}}}

%%ZK VPD
\newcommand{\zkVPDKeyGen}{\texttt{zkVPD.KeyGen}}
\newcommand{\zkVPDCommit}{\texttt{zkVPD.Commit}}
\newcommand{\zkVPDOpen}{\texttt{zkVPD.Open}}
\newcommand{\zkVPDVerify}{\texttt{zkVPD.Verify}}

%%Algorithms Linear Time Prover
\newcommand{\Nx}{\cN_\vx}
\newcommand{\Ny}{\cN_\vy}
\newcommand{\Precomp}[1]{\texttt{Precompute}(#1)}
\newcommand{\LUT}[1]{\ensuremath{\boldsymbol{T}_{#1}}}
\newcommand{\Initone}[3]{\texttt{Initialize\_Phase\_One}(\ensuremath{#1},\ensuremath{#2},\LUT{#2},\ensuremath{\makevector{#3}})}
\newcommand{\Initonecomb}[6]{\texttt{Initialize\_Phase\_One\_Combined}(\ensuremath{#1},\ensuremath{#2},\LUT{#2},\ensuremath{\makevector{#3}},\ensuremath{\makevector{#4}},\ensuremath{#5},\ensuremath{#6})}
\newcommand{\Inittwo}[3]{\texttt{Initialize\_Phase\_Two}(\ensuremath{#1},\ensuremath{#2},\ensuremath{#3})}
\newcommand{\FuncEval}[4]{\texttt{Function\_Evaluations}(\ensuremath{#1},\LUT{#2},\ensuremath{#3},\ldots,\ensuremath{#4})}
\newcommand{\Sumcheck}[3]{\texttt{Sumcheck}(\ensuremath{#1},\LUT{#1},\ensuremath{#2},\ldots,\ensuremath{#3})}
\newcommand{\SumcheckProd}[6]{\texttt{Sumcheck\_Product}(\ensuremath{#1},\LUT{#2},\ensuremath{#3},\LUT{#4},\ensuremath{#5},\ldots,\ensuremath{#6})}
\newcommand{\SumcheckProdFinal}[6]{\texttt{Sumcheck\_Product}(\ensuremath{#1},\ensuremath{#2},\ensuremath{#3},\ensuremath{#4},\ensuremath{#5},\ldots,\ensuremath{#6})}
\newcommand{\LinGKR}[6]{\texttt{Linear\_Prover\_GKR}(\ensuremath{#1},\ensuremath{#2},\ensuremath{#3},\ensuremath{#4},\ensuremath{#5},\ensuremath{#6})}


\mathchardef\mhyphen="2D

% Box environment
% Usage: \begin{Boxfig}[placement]{Caption}{Label}{Title}
\newenvironment{Boxfig}[4][]{
    \begin{figure}[#1]
    \newcommand{\FigCaption}{#2}
    \newcommand{\FigLabel}{#3}
    \vspace{-0.25cm}
    \begin{center}
        \begin{small}
        \begin{tabular}{@{}|@{~~}l@{~~}|@{}}
        \hline
        \rule[-1.5ex]{0pt}{1ex}\begin{minipage}[b]{.96\linewidth}
        \vspace{1ex}
        \smallskip
        \begin{center}
        \textbf{#4}
        \end{center}
}
{%
        \end{minipage}\\
        \hline
        \end{tabular}
        \end{small}
        \vspace{-0.25cm}
        \caption{\FigCaption}
        \label{fig:\FigLabel}
    \end{center}
  \vspace{-.5cm}
    \end{figure}
}

\mathchardef\mhyphen="2D

\newtheorem{observation}{Observation}

%\newtheorem{theorem}{Theorem}[section]
%\newtheorem{corollary}{Corollary}[theorem]
%\newtheorem{lemma}[theorem]{Lemma}
%\newtheorem{claim}[theorem]{Claim}




\newcommand{\asn}{:=}
\newcommand{\sample}{\leftarrow}

% Various brackets
\newcommand{\round}[1]{\lceil #1 \rfloor}
\newcommand{\bround}[1]{\left\lfloor #1 \right\rceil}
\newcommand{\ceil}[1]{\lceil #1 \rceil}
\newcommand{\floor}[1]{\lfloor #1 \rfloor}
\newcommand{\abs}[1]{\lvert #1 \rvert}
%\newcommand{\ang}[1]{\langle #1 \rangle}
\newcommand{\sh}[1]{[#1]}
\newcommand{\shl}[1]{\langle #1 \rangle}
%\newcommand{\shL}[1]{\sh[L]{#1}}
%\newcommand{\shR}[1]{\sh[R]{#1}}
\newcommand{\shcomm}[1]{\llbracket #1 \rrbracket}
\newcommand{\shG}[1]{[#1]_{\GG}}
\newcommand{\share}[1]{[ #1 ]}
\newcommand{\fshare}[1]{[ #1 ]_{\FF}}
\newcommand{\rshare}[1]{[ #1 ]_{R}}


%%% Encodings
\newcommand{\CAFE}{\text{Circuit Amortization Friendly Encodings}}
\newcommand\Enc{\ensuremath{\mathsf{E}}}
\newcommand\InnerProd{\ensuremath{\mathsf{InnerProd}}}
\newcommand\einL{\ensuremath{\Enc_{L}}}
\newcommand\einR{\ensuremath{\Enc_{R}}}
\newcommand\eout{\ensuremath{\Enc_{out}}}



% Polynomial/negligible functions
\newcommand{\poly}{\mathsf{poly}}
\newcommand{\negl}{\mathsf{negl}}

%%%
%\newcommand{\F}{\mathbb{F}}
\newcommand{\bN}{\mathbb{N}}
\newcommand{\alg}{\F}
\newcommand{\secpar}{\kappa}
\newcommand{\groupgen}{\mathsf{GroupGen}}
\newcommand{\KEnc}{\mathsf{Gen}}
\newcommand{\enc}{\texttt{Encrypt}}
\newcommand{\wenc}{\tilde{\mathsf{E}}}
\newcommand{\dec}{\texttt{Decrypt}}
\newcommand{\KeyGen}{\texttt{KeyGen}}
\newcommand{\setup}{\mathsf{Setup}}
\newcommand{\prove}{\mathsf{Prove}}
\newcommand{\verify}{\mathsf{Verify}}
\newcommand{\eval}{\mathsf{Eval}}
\newcommand{\Rel}{\mathcal{R}}
\newcommand{\ext}{\chi}
\newcommand{\simu}{\mathcal{S}}
\newcommand{\st}{\mathsf{state}}
\newcommand{\pk}{\texttt{pk}}
\newcommand{\sk}{\texttt{sk}}
\newcommand{\vk}{\mathsf{vk}}
\newcommand{\td}{\mathsf{td}}
\newcommand{\crs}{\mathsf{crs}}
\newcommand{\priv}{\mathsf{priv}}
\newcommand{\scrs}{\mathsf{vcrs}}
\newcommand{\vpoly}{v_{\mathsf{poly}}}
\newcommand{\wpoly}{w_{\mathsf{poly}}}
\newcommand{\ypoly}{y_{\mathsf{poly}}}
\newcommand{\norm}[1]{\left\lVert#1\right\rVert}

%bold fonts
\newcommand{\bG}{\mathbb{G}}
\newcommand{\bGT}{\mathbb{G}_T}

%%%%% VC%%
\newcommand{\kgen}{\mathsf{KGen}}
\newcommand{\probgen}{\mathsf{ProbGen}}
\newcommand{\compute}{\mathsf{Compute}}
\newcommand{\ver}{\mathsf{Ver}}
% Algorithms

\newcommand{\he}{\mathsf{HE}}
\newcommand{\Decode}{\mathsf{Decode}}



\newcommand{\bool}{\{0,1\}}
\newcommand{\expVC}{\mathsf{Expt}_{\mathcal{A}}^{Ver}}
% Vectors
\newcommand{\makevector}[1]{{\ensuremath{\boldsymbol{#1}}}}
%\newcommand{\makevector}[1]{{\ensuremath{\vec{#1}}}}
\newcommand{\va}{\textbf{a}}
\newcommand{\vb}{\textbf{b}}
\newcommand{\vc}{\textbf{c}}
\newcommand{\ve}{\textbf{e}}
\newcommand{\vf}{\textbf{f}}
\newcommand{\vg}{\textbf{g}}
\newcommand{\vh}{\textbf{h}}
\newcommand{\vm}{\textbf{m}}
\newcommand{\vq}{\makevector{q}}
\newcommand{\vp}{\makevector{p}}
\newcommand{\vr}{\makevector{r}}
\newcommand{\vs}{\makevector{s}}
\newcommand{\vt}{\makevector{t}}
\newcommand{\vTT}{\makevector{\TT}}
\newcommand{\vu}{\makevector{u}}
\newcommand{\vUU}{\makevector{\UU}}
\newcommand{\vv}{\makevector{v}}
\newcommand{\vw}{\makevector{w}}
\newcommand{\vWW}{\makevector{\WW}}
\newcommand{\vx}{\makevector{x}}
\newcommand{\vXX}{\makevector{\XX}}
\newcommand{\vvXX}{\XX_1, \ldots, \XX_n}
\newcommand{\vYY}{\makevector{\YY}}
\newcommand{\vZvar}{\makevector{\Zvar}}
\newcommand{\vy}{\makevector{y}}
\newcommand{\vz}{\makevector{z}}
\newcommand{\vdelta}{\makevector{\delta}}
\newcommand{\vchi}{\makevector{\chi}}
\newcommand{\vpsi}{\makevector{\psi}}

% Polynomial-related
% \newcommand{\EvR}[2]{\Epsilon_{#2}(#1)}
% \newcommand{\EvL}[2]{{}_{#2}\Epsilon(#1)}
\newcommand{\Ev}[2]{\mathtt{Ev}_{#2}(#1)}
\newcommand{\EvR}[2]{{#1}^{\textsf{R}}(#2)}
\newcommand{\EvL}[2]{{#1}^{\textsf{L}}(#2)}
\newcommand{\UU}{\mathtt{U}}
\newcommand{\WW}{\mathtt{W}}
\newcommand{\XX}{\mathtt{X}}
\newcommand{\YY}{\mathtt{Y}}
\newcommand{\Zvar}{\mathtt{Z}}
\newcommand{\TT}{\mathtt{T}}


% Matrices

\newcommand{\makematrix}[1]{{\ensuremath{\mathbf{#1}}}}
\newcommand{\Diag}{\mathsf{Diag}}
\newcommand{\matA}{\makematrix{A}}
\newcommand{\mM}{\makematrix{M}}
%\newcommand{\mC}{\makematrix{C}}
\newcommand{\mD}{\makematrix{D}}
%\newcommand{\mH}{\makematrix{h}}
%\newcommand{\mQ}{\makematrix{q}}
%\newcommand{\mT}{\makematrix{t}}
%\newcommand{\mZ}{\makematrix{z}}
%\newcommand{\mH}{\makevector{H}}
\newcommand{\mQ}{\makevector{Q}}
\newcommand{\mT}{\makevector{T}}
\newcommand{\mZ}{\makevector{Z}}
\newcommand{\mOne}{\makematrix{1}}
\newcommand{\mZero}{\makematrix{0}}


\newcommand{\PRF}{\mathsf{PRF}}
\newcommand{\PRG}{\mathsf{PRG}}
\newcommand{\hash}{\mathsf{H}}

\newcommand{\OT}{{\mathsf{OT}}}

\newcommand{\PSOne}{\mathcal{C}}



% Adversaries etc.
\newcommand{\Env}{\mathcal{Z}}
\newcommand{\Sim}{\mathcal{S}}
\newcommand{\Adv}{\mathcal{A}}
%\newcommand{\A}{\mathcal{A}}
\newcommand{\adv}{\mathcal{A}}
\newcommand{\bdv}{\mathcal{B}}
\renewcommand{\ext}{{\chi}}
%\newcommand{\ct}{\mathsf{C}}
\newcommand{\evk}{\texttt{evk}}
\newcommand{\keyswitch}{\texttt{KeySwitch}}
\newcommand{\premult}{\texttt{Pre-Multiply}}
%\newcommand{\scale}{\texttt{Rescale}}
\newcommand{\rings}{\mathcal{S}}
\newcommand{\plushe}{+_{\text{HE}}}
\newcommand{\plushei}{+_{\text{HE}_i}}
\newcommand{\pluslev}[1]{+_{\text{HE}_{#1}}}
\newcommand{\timeshe}{\times_{\text{HE}}}
\newcommand{\timeshei}{\times_{\text{HE}_i}}
\newcommand{\timeslev}[1]{\times_{\text{HE}_{#1}}}

\newcommand{\AddMAC}{\mathsf{AddMAC}}
\newcommand{\InputMAC}{\mathsf{MACPrivateInput}}
\newcommand{\AddMACT}{2\mhyphen \mathsf{AddMAC}}
\newcommand{\AddMACn}{n\mhyphen \mathsf{Bracket}}
%\newcommand{\Add}{\mathsf{Add}}


% Protocols
\newcommand{\Prot}[1]{\ensuremath{\Pi_{\scriptstyle\mathrm{#1}}}}
\newcommand{\PPrep}{\Prot{Prep}}
\newcommand{\PPrepCheck}{\Prot{PrepCheck}}
\newcommand{\PRand}{\Prot{Rand}}
\newcommand{\PMPC}{\Prot{MPC}}
\newcommand{\POpen}{\Prot{Open}}
\newcommand{\Pshcomm}{\Prot{\shcomm{\cdot}}}
\newcommand{\Psh}{\Prot{\sh{\cdot}}}
\newcommand{\Pshl}{\Prot{\shl{\cdot}}}

% Functionalities
\newcommand{\Func}[1]{\ensuremath{\mathcal{F}_{\scriptstyle\mathrm{#1}}}}
\newcommand{\Fauth}{\Func{\spdz{\cdot}}}
\newcommand{\FCT}{\Func{CoinToss}}
\newcommand{\Fthree}[1]{\Func{3PC}(#1)}
\newcommand{\Ftriples}{\Func{Triples}}


% Commands for functionalities
\newcommand{\FCommand}[1]{\ensuremath{\mathtt{#1}}}
\newcommand{\id}{\FCommand{id}}
\newcommand{\Val}[1]{\FCommand{Val}[#1]}
\newcommand{\Input}{\FCommand{Input}}
\newcommand{\Output}{\FCommand{Output}}
\newcommand{\Triple}{\FCommand{Triple}}
%\newcommand{\Mult}{\FCommand{Multiply}}
\newcommand{\coin}{\FCommand{Toss}}
\renewcommand{\check}{\FCommand{Check}}
\newcommand{\commit}{\FCommand{Commit}}
\newcommand{\ok}{\FCommand{ok}}
%\newcommand{\GO}{\mathit{GO}}
%\newcommand{\NOGO}{\mathit{NO{\mhyphen}GO}}
\newcommand{\Prep}{\FCommand{Prep}}


%
% Referencing
%
\newcommand{\appref}[1]{Appendix~\ref{app:#1}}
\newcommand{\lemmaref}[1]{Lemma~\ref{lem:#1}}
\newcommand{\figref}[1]{Figure~\ref{fig:#1}}
\renewcommand{\eqref}[1]{(\ref{eq:#1})}
\newcommand{\tabref}[1]{Table~\ref{tab:#1}}
\newcommand{\secref}[1]{Section~\ref{sec:#1}}
\newcommand{\protref}[1]{Protocol~\ref{alg:#1}}
\newcommand{\figlab}[1]{\label{fig:#1}}

% use this to number a single line of align* equations
\newcommand\numberthis{\addtocounter{equation}{1}\tag{\theequation}}

\newenvironment{boxfig}[2]{% {#1}{#2} = {Caption}{label}
     \begin{figure}
     \newcommand{\FigCaption}{#1}
     \newcommand{\FigLabel}{#2}
       \vspace{-.30cm}
     \begin{center}
       \begin{small}
         \begin{tabular}{@{}|@{~~}l@{~~}|@{}}
           \hline
           %\rule[-1ex]{0pt}{1ex}\begin{minipage}[!htb]{\textwidth}
            \rule[-1.5ex]{0pt}{1ex}\begin{minipage}[b]{.96\linewidth}
             \vspace{1ex}
             \smallskip
             }{%
           \end{minipage}\\
           \hline
         \end{tabular}
       \end{small}
        \vspace{-0.25cm}
       \caption{\FigCaption}
       \figlab{\FigLabel}
     \end{center}
     \vspace{-.2cm}
   \end{figure}
}

%%%%%%%%%%%%%%%%%%%%%%%%%%%%
%    Caligraphic Alphabet  %
%%%%%%%%%%%%%%%%%%%%%%%%%%%%
\newcommand{\cA}{{\cal A}}
\newcommand{\cB}{{\cal B}}
\newcommand{\cC}{{\cal C}}
\newcommand{\cD}{{\cal D}}
\newcommand{\cE}{{\cal E}}
\newcommand{\cF}{{\cal F}}
\newcommand{\cG}{{\cal G}}
\newcommand{\cH}{{\cal H}}
\newcommand{\cI}{{\cal I}}
\newcommand{\cK}{{\cal K}}
\newcommand{\cL}{{\cal L}}
\newcommand{\cM}{{\cal M}}
\newcommand{\cN}{{\cal N}}
\newcommand{\cO}{{\cal O}}
\newcommand{\cP}{{\cal P}}
\newcommand{\cQ}{{\cal Q}}
\newcommand{\cR}{{\cal R}}
\newcommand{\cS}{{\cal S}}
\newcommand{\cT}{{\cal T}}
\newcommand{\cU}{{\cal U}}
\newcommand{\cV}{{\cal V}}
\newcommand{\cX}{{\cal X}}
\newcommand{\cY}{{\cal Y}}
\newcommand{\cZ}{{\cal Z}}
\newcommand{\cW}{{\cal W}}

\newcommand{\Gen}{{\cal G}}
\newcommand{\Prover}{{\cal P}}
\newcommand{\Verifier}{{\cal V}}

%%%%%%%%%%%%%%%%%%%%%%%%%%%%
% Blackboard Bold Alphabet %
%%%%%%%%%%%%%%%%%%%%%%%%%%%%
\renewcommand{\AA}{\mathbf{A}}
\newcommand{\BB}{\mathbb{B}}
\newcommand{\CC}{\mathbb{C}}
\newcommand{\E}{\mathbb{E}}
\newcommand{\FF}{\mathbb{F}}
\newcommand{\GG}{\mathbb{G}}
\newcommand{\HH}{\mathbb{H}}
\newcommand{\KK}{\mathbb{K}}
\newcommand{\NN}{\mathbb{N}}
\newcommand{\QQ}{\mathbb{Q}}
\newcommand{\RR}{\mathbb{R}}
%\newcommand{\TT}{\mathbb{T}}
\newcommand{\ZZ}{\mathbb{Z}}

%%%%%%%%%%%%%%%%%%%%%%%%%%
% Boldface Math Alphabet %
%%%%%%%%%%%%%%%%%%%%%%%%%%
\def \bfa{{\mathbf a}}
\def \bfb{{\mathbf b}}
\def \bfc{{\mathbf c}}
\def \bfe{{\mathbf e}}
\def \bff{{\mathbf f}}
\def \bfg{{\mathbf g}}
\def \bfm{{\mathbf m}}
\def \bfp{{\mathbf p}}
\def \bfr{{\mathbf r}}
\def \bfs{{\mathbf s}}
\def \bft{{\mathbf t}}
\def \bfu{{\mathbf u}}
\def \bfv{{\mathbf v}}
\def \bfw{{\mathbf w}}
\def \bfx{{\mathbf x}}
\def \bfy{{\mathbf y}}
\def \bfz{{\mathbf z}}
\def \bfA{{\mathbf A}}
\def \bfH{{\mathbf H}}
\def \bfZ{{\mathbf Z}}
\def \bfS{{\mathbf S}}
\def \wb{{\widehat{b}}}

% Some specific rings
\newcommand{\GR}{\mathsf{GR}}
\newcommand{\Ztwok}{\ensuremath{\mathbb{Z}_{2^k}}}
\newcommand{\MatRing}[3]{\mathcal{M}_{#1 \times #2}(#3)}
\newcommand{\SqMat}[1]{\mathcal{M}_{#1 \times #1}}
\newcommand{\Mn}[1]{\mathcal{M}_{n \times n}(#1)}
\newcommand{\Mparties}[1]{\mathcal{M}_{n \times n}(#1)}
\newcommand{\MnR}{\Mn{R}}
\newcommand{\MZtwok}{\Mn{\Ztwok}}
\newcommand{\Mthree}[1]{\mathcal{M}_{3 \times 3}(#1)}
\newcommand{\MthreeR}{\Mthree{R}}

%FHE-land
%\newcommand{\sk}{\texttt{sk}}
%\newcommand{\pk}{\texttt{pk}}
\newcommand{\ct}{\texttt{ct}}
\newcommand{\ksd}{\texttt{ksd}}
\newcommand{\erfc}{\texttt{erfc}}
\newcommand{\erf}{\texttt{erf}}
\newcommand{\erfinv}{\texttt{erf}^{-1}}
\newcommand{\lev}{t}
\newcommand{\ql}{{q_{\lev}}}
\newcommand{\BGV}{^\mathsf{BGV}}
\newcommand{\FV}{^\mathsf{FV}}
\newcommand{\CRT}{\mathsf{CRT}}
\newcommand{\clean}{\mathsf{clean}}
\newcommand{\ReduceLevel}{{\mathsf{ReduceLevel}}}
\newcommand{\scale}{{\texttt{Scale}}} 
\newcommand{\scalenew}{{\mathsf{Scale_{VC}}}}
\newcommand{\realmodred}{{\mathsf{RealModRed}}} 
\newcommand{\SwitchKey}{\mathsf{SwitchKey}}
\newcommand{\SwitchKeyGen}{\mathsf{SwitchKeyGen}}
%\newcommand{\asn}{\leftarrow}
%\newcommand{\cK}{c_m}
\newcommand{\can}{\mathsf{can}}
\newcommand{\phim}{\phi(m)}
\newcommand{\BKs}[3]{B_\mathsf{Ks,#2}#1(#3)}
\newcommand{\dt}{\mathfrak{d}}
\newcommand{\Add}{\mathsf{Add}}
\newcommand{\Mult}{\mathsf{Mult}}
%\newcommand{\Decode}{\mathsf{Decode}}
\newcommand{\Encr}{\ensuremath{\texttt{Enc}}}
\newcommand{\Decr}{\ensuremath{\texttt{Dec}}}
\newcommand{\Eval}{\ensuremath{\texttt{Eval}}}
\newcommand{\ring}{\mathcal{R}}
\newcommand{\realmod}[1]{\RR_{[-#1, #1)}}
\newcommand{\newring}{\mathcal{S}}

%Distributions
\newcommand{\calU}{\mathcal{U}}
\newcommand{\calD}{\mathcal{D}}
\newcommand{\calN}{\mathcal{N}}
\newcommand{\dN}{\mathcal{DG}}
\newcommand{\HWT}{\mathcal{HWT}}
\newcommand{\ZO}{\mathcal{ZO}}


%\newtheorem{thm}{Theorem}%[section]      % A counter for Theorems etc
\newcommand{\BT}{\begin{thm}}   \newcommand{\ET}{\end{thm}}
%\newtheorem{dfn}[thm]{Definition}      %
\newcommand{\BD}{\begin{dfn}}   \newcommand{\ED}{\end{dfn}}
%\newtheorem{cor}[thm]{Corollary}      %
\newcommand{\BCR}{\begin{cor}} \newcommand{\ECR}{\end{cor}}
%---
%\newtheorem{Ithm}{Theorem}[section]     % A counter for Theorems in Intro
\newcommand{\BIT}{\begin{Ithm}}   \newcommand{\EIT}{\end{Ithm}}
%---
%\newtheorem{lem}{Lemma}%[section]  % A counter for Lemmas etc
\newcommand{\BL}{\begin{lem}}   \newcommand{\EL}{\end{lem}}
%\newtheorem{prop}[lem]{Proposition}
\newcommand{\BP}{\begin{prop}}   \newcommand{\EP}{\end{prop}}
%\newtheorem{clm}[lem]{Claim}            %
\newcommand{\BCM}{\begin{clm}}   \newcommand{\ECM}{\end{clm}}
%\newtheorem{fact}[lem]{Fact}            %
\newcommand{\BF}{\begin{fact}}   \newcommand{\EF}{\end{fact}}
%\newenvironment{proof}{\noindent{\bf Proof:~~}}{\qed}
\newcommand{\BPF}{\begin{proof}} \newcommand {\EPF}{\end{proof}}
\newtheorem{prot}{Protocol}      % A counter for Protocols
\newenvironment{sproof}{\noindent{\bf Proof Sketch:~~}}{\qed}
\newcommand{\BPFS}{\begin{sproof}} \newcommand {\EPFS}{\end{sproof}}
\newcommand{\BPR}{\begin{prot}}   \newcommand{\EPR}{\end{prot}}
\newenvironment{cproof}{\noindent{\bf Proof:~~}}{\hfill $\Box$}
\newcommand{\BCPF}{\begin{cproof}} \newcommand {\ECPF}{\end{cproof}}
\newcommand{\Qed}{\hfill $\Box$}

\newcommand{\BDE}{\begin{description}}
\newcommand{\EDE}{\end{description}}
\newcommand{\BE}{\begin{enumerate}}
\newcommand{\EE}{\end{enumerate}}
\newcommand{\BI}{\begin{itemize}}
\newcommand{\EI}{\end{itemize}}
\newcommand{\BEQ}{\begin{eqnarray*}}
\newcommand{\EEQ}{\end{eqnarray*}}
\def\blackslug
{\hbox{\hskip 1pt\vrule width 8pt height 8pt depth 1.5pt\hskip 1pt}}
\def\qed{\quad\blackslug\lower 8.5pt\null\par}


\newcommand{\defeq}{\coloneqq}
\newcommand{\indist}{\stackrel{\rm c}{\approx}}
\newcommand{\prob}{{\rm Pr}}
\newcommand{\bits}{\{0,1\}}
\newcommand{\ppt}{{\sc PPT}}
\newcommand{\view}{\bf View}
\newcommand{\ideall}{\mathbf{IDEAL}}
\newcommand{\hyb}{{\mathbf{HYB}}}
\newcommand{\Hyb}{{\mathbf{H}}}
\newcommand{\reall}{\mathbf{REAL}}
\newcommand{\wreall}{\widetilde{\mathbf{REAL}}}
\newcommand{\whyb}{\widetilde{\mathbf{HYB}}}
\newcommand{\wideall}{\widetilde{\mathbf{IDEAL}}}
\newcommand{\circc}{\mathsf{Circ}}
\newcommand{\rand}{\mathsf{Rand}}
\newcommand{\flip}{\mathsf{Flip}}
\newcommand{\guess}{\mathsf{Guess}}
\newcommand{\brho}{\boldsymbol{\rho}}
\newcommand{\wsim}{\widetilde{\Sim}}

%%%%%


\newtheorem{construction}{Construction}{\bfseries}{\rmfamily}


%\newcommand{\F}{\mathbb{F}}                
\newcommand{\Z}{\mathbb{Z}}                

\renewcommand{\O}{\operatorname{\mathcal{O}}}   % big-O notation

%%Eduardo: Note the \renewcommand for \prot
\renewcommand{\prot}[1]{\ensuremath{\mathsf{#1}}\xspace}
\newcommand{\Share}{\prot{Share}}
\newcommand{\ShareConv}{\prot{ShareConversion}}
\newcommand{\Rec}{\prot{ReconsPriv}}
\newcommand{\RecP}{\prot{ReconsPubl}}
\newcommand{\RecA}{\prot{ReconsAsync}}
\newcommand{\DoubleSh}{\prot{DoubleShareRandom}}
\newcommand{\PrepPhase}{\prot{PreparationPhase}}
\newcommand{\CompPhase}{\prot{ComputationPhase}}
\newcommand{\GiveInput}{\prot{GiveInput}}
\newcommand{\InputPhase}{\prot{InputPhase}}
\newcommand{\CheckAlive}{\prot{CheckAlive}}
\newcommand{\BCU}{\prot{Broadcast}}


\newenvironment{mylist}[1][$\bullet$]%
{\begin{list}{$\bullet$}{\labelwidth=5cm\settowidth{\leftmargin}{#1}%
			\addtolength{\leftmargin}{.8em}\itemsep=0pt\parsep=.5ex\topsep=0pt}}%
	{\end{list}}

\long\def\protocolbeg#1#2#3\item{%
	\paragraph*{\normalfont\bfseries Protocol #2.}
	\vspace*{0pt}\mbox{}\par\noindent\ignorespaces #3 \ignorespaces%
	\begin{list}{$\bullet$}{\labelwidth=5cm\settowidth{\leftmargin}{#1}%
			\addtolength{\leftmargin}{.8em}\itemsep=0pt\parsep=.5ex\topsep=0pt%
			\partopsep=0pt}%
		\item}
	\newenvironment{protocol}[2][$\bullet$]{\protocolbeg{#1}{#2}}%
	{\end{list}\medskip}

\newenvironment{costs}{\par\noindent\emph{Costs:}}{\par}

%%%%Actual new commands for this paper
\newcommand{\size}[1]{|#1|}
\newcommand\restr[2]{{% we make the whole thing an ordinary symbol
		\left.\kern-\nulldelimiterspace % automatically resize the bar with \right
		#1 % the function
		\vphantom{\big|} % pretend it's a little taller at normal size
		\right|_{#2} % this is the delimiter
}}
%%% Local Variables:
%%% mode: latex
%%% TeX-master: "main"
%%% End:

%\usepackage{xparse}

\NewDocumentCommand{\INTERVALINNARDS}{ m m }{
	#1 {,} #2
}
\NewDocumentCommand{\interval}{ s m >{\SplitArgument{1}{,}}m m o }{
	\IfBooleanTF{#1}{
		\left#2 \INTERVALINNARDS #3 \right#4
	}{
		\IfValueTF{#5}{
			#5{#2} \INTERVALINNARDS #3 #5{#4}
		}{
			#2 \INTERVALINNARDS #3 #4
		}
	}
}

